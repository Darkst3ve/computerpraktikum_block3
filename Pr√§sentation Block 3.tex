\documentclass{beamer}
\usepackage[T1]{fontenc}
\usepackage[utf8]{inputenc}
\usepackage[ngerman,english,dutch,strings]{babel}
\usepackage{latexsym} 
\usepackage{amsfonts} 
\usepackage{amsmath}
\usepackage{amssymb}
\usepackage{amsthm}
\usepackage{bbm}

\AtBeginSection[]{
  \begin{frame}
  \vfill
  \centering
  \begin{beamercolorbox}[sep=8pt,center,shadow=true,rounded=true]{title}
    \usebeamerfont{title}\insertsectionhead\par%
  \end{beamercolorbox}
  \vfill
  \end{frame}
}

\title{Binäre Klassifikation mit Python}
\author{Lennart Duvenbeck, Adrian Schoch, Markus Duong}
\date{09.07.2019}

\begin{document}
\maketitle
\begin{frame}{Inhaltsverzeichnis}
  \tableofcontents
\end{frame}

\section{Aufgabenstellung und Verfahren}
\begin{frame} %%Eine Folie
\frametitle{Aufgabenstellung} %%Folientitel
Mit Hilfe eines Datensatzes $D = ((y_{1},x_{1}),...,(y_{n},x_{n}))$, wobei $y_{i} \in \{-1,1\}$ und $x_{i} \in [-1,1]^{d}$ gilt, soll eine Funktion $f_{D}: [-1,1]^{d} \rightarrow \{-1,1\}$ gefunden werden, so dass für einen vorher nicht gesehenen Datensatz $D' = ((x_{1},y_{1}),...,(x_{n},y_{n}))$ die Fehlerklassifikationsrate 
\[
\mathcal{R}_{D'}(f_{D}) := \frac{1}{m} \sum_{i=1}^{m} \mathbbm{1}_{y_{i}' \neq f_{D}(x_{i}')}
\]
möglichst gering ist.
\end{frame}

\begin{frame} %%Eine Folie
\frametitle{Verfahren} %%Folientitel
\textbf{Schritt 1:} Zu einem gegebenen Parameter $k \in \mathbb{N}$, einem Datensatz $D$ und einem zu klassifizierenden Punkt $x \in [-1,1]^{d}$ werden zunächst die $k$-nächsten Nachbarn $x_{i_{1}},...,x_{i_{k}}$ von $x$ in $D$ identifiziert. \\
\ \\
\textbf{Schritt 2:} Berechne
\[
f_{D,k}(x) := \text{sign} \bigg( \sum_{j =1}^{k} y_{i_{j}}\bigg).
\]
Hierbei ist $\text{sign} \: 0 := 1$.
\end{frame}

\begin{frame} %%Eine Folie
\frametitle{Verfahren} %%Folientitel
\textbf{Schritt 3:}  Um das \glqq optimale\grqq{} $k$ zu bestimmen wird der Datensatz $D$ zufällig in $l$ gleichgroße Teildatensätze $D_{1},...,D_{l}$ zerlegt und eine endliche Menge $K$ möglicher Werte für $k$ betrachtet.\\
\ \\
\textbf{Schritt 4:} Für $k \in K$, $i = 1,...,l$ und $D_{\textbackslash i} := D_{1} \cup ... \cup D_{i-1} \cup D_{i+1} \cup...\cup D_{l}$ wird dann
\[
\mathcal{R}_{D_{i}}(f_{D_{\textbackslash i}, k})
\]
berechnet und anschließend der Mittelwert über $i = 1,...,l$ gebildet. Ist $k^{*}$ das $k$ in $K$ mit dem kleinsten Mittelwert, so ist der resultierende Klassifikator
\[
f_ {D} := \text{sign} \bigg( \sum_{i=1}^{l} f_{D_{\textbackslash i}, k^{*}}\bigg).
\]
\end{frame}
\section{Implementierung}
\section{Interessanter Zwischenergebnisse und ihre graphische Darstellung}
\end{document}
